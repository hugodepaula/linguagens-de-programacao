%%%%%%%%%%%%%%%%%%%%%%%%%%%%%%%%%%%%%%%
% Cronograma de aula
%%%%%%%%%%%%%%%%%%%%%%%%%%%%%%%%%%%%%%%

\documentclass[11pt,brazil, a4paper, fullpage]{article}


\newcommand\UNIVERSIDADE{PONTIFÍCIA  UNIVERSIDADE  CATÓLICA  DE  MINAS  GERAIS}
\newcommand\UNIDADE{Praça da Liberdade}
\newcommand\INSTITUTO{Inst. de Ciências Exatas e Informática}
\newcommand\CURSO{Ciência da Computação}
\newcommand\PROFESSOR{Hugo de Paula}
\newcommand\EMAILPROFESSOR{hugo@pucminas.br}
\newcommand\DEPARTAMENTO{Departamento de Ciência da Computação}

\newcommand\DISCIPLINA{Linguages de Programação}
\newcommand\ANO{2024}
\newcommand\SEMESTRE{1}
\newcommand\TURNO{manhã}
\newcommand\PERIODO{3}

\def\year{\ANO}

\input{infos/some-definitions}
\input{infos/calpreambulo}

\newcommand{\DIAUM}{\Segunda}
\newcommand{\DIADOIS}{\Quarta}


\begin{document}

	\selectlanguage{brazil}

	\logoPUC{infos/puclogo_small_bw}
	\dadosDisciplina{\DISCIPLINA}{\CURSO}{\TURNO}{\ANO}{\SEMESTRE}{\PROFESSOR}{\EMAILPROFESSOR}

	%\dadosAvancadosDisciplina{04}{00}{04}{80}{04}{infos/objetivos.tex}{infos/ementa.tex}{3 $\times$ 25}{25}{infos/avaliacao.tex}{infos/biblio.tex}

	\calVisualColorido{1}{\DIAUM}{\DIADOIS}{\ANO}{07-07} % semestre, dias de aula (DOIS vezes por semana)
	%\calVisualColorido{2}{\DIAUM}{\DIADOIS}{\ANO}{12-21} % semestre, dias de aula (DOIS vezes por semana)


\begin{center}

\small
%\footnotesize
\begin{calendar}{2/5/\ANO}{22} % Semestre comeca no dia 1 de agosto, e dura por 21 semanas.
\setlength{\calboxdepth}{.3in}
\setlength{\calwidth}{0.95\textwidth}

%
%% configuracao da semana
\semanaSegQua
%
%% lista de Aulas
\caltexton{1}{Apresentação do curso, domínios de programação, sintaxe e semântica. (Cap. 1)}
\caltextnext{Principais paradigmas de linguagens de programação: imperativo, funcional e lógico. (material próprio)}
%\caltextnext{Evolução das principais linguagens de programação. (Cap. 2)}
%\caltextnext{Sintaxe e semântica (Cap. 3 e 4)}
\caltextnext{Nomes e declaração. Associação: estática, dinâmica, declaração. (Cap. 5)}
\caltextnext{Nomes e declaração. Associação: estática, dinâmica, declaração. (Cap. 5)}
\caltextnext{Variáveis: atualização, escopo e tempo de vida. (Cap. 5)}
\caltextnext{Variáveis estáticas. Coletores de Lixo (Cap. 5)}
\caltextnext{Tipos de dados. Tipos primitivos. (Cap. 6)}
\caltextnext{Tipos primitivos enumerados e subfaixas. Tipos compostos: produto cartesiano. (Cap. 6)}
\caltextnext{Tipos compostos: união disjunta, mapeamento e conjunto potência. (Cap. 6)}
%\caltextnext{Tipos persistentes e transientes, ponteiros e referências. (Cap. 6)}
\caltextnext{Sistemas de Tipos. Tipagem estática e dinâmica. Equivalência de tipos: estrutural e por nomes. (Cap. 6)}
%\caltextnext{Tipagem forte.  (Cap. 6)}
\caltextnext{Expressões aritméticas. Sobrecarga de operadores. Conversão de tipos.  (Cap. 7)}
\caltextnext{Comandos de atribuição. Atribuição mista. (Cap. 7) Instruções de controle. (Cap. 8)}
\exercicios{Aula de Exercícios}
\prova{Prova 1}
%\caltextnext{Instruções de controle. (Cap. 8)}
\caltextnext{Subprogramas e passagem de parâmetros: mecanismos de cópia. (Cap. 9)}
\caltextnext{Subprogramas e passagem de parâmetros: mecanismos de definição. (Cap. 9)}
\caltextnext{Subprogramas genéricos. Sobrecarga de sub-programas, fechamentos e Co-rotinas. (Cap. 9)}
\caltextnext{Subprogramas genéricos. Sobrecarga de sub-programas. (Cap. 9)}
\caltextnext{Fechamentos e Corotinas. (Cap. 9)}
%\caltextnext{Abstração de dados e encapsulamento. (Cap. 11)}
%\caltextnext{Encapsulamento. (Cap. 11)}
\caltextnext{Princípios da programação orientada para objetos: atributos, métodos, construtores, destrutores. (Cap. 12)}
\caltextnext{Hierarquias de tipos e de classes. Atributos estáticos na programação orientada para objetos. (Cap. 12)}
\caltextnext{Polimorfismo de inclusão. (Cap. 12)}
\caltextnext{Polimorfismo paramétrico. (Cap. 12)}
\caltextnext{Coleções. Funções Lambda e referência de métodos. (Cap. 12)}
\exercicios{Aula de exercícios}
\prova{Prova 2}
%\caltextnext{Polimorfismo paramétrico. (Cap. 12)}
\caltextnext{Concorrência: Threads (Cap. 13)}
\caltextnext{Concorrência: Locks e sincronização (Cap. 13)}
\caltextnext{Concorrência: Thread pool (Cap. 13)}
%\caltextnext{Tratamento de Exceções. (Cap. 14)}
%\caltextnext{Orientação a eventos. (Cap. 14)}
\caltextnext{Programação funcional: funções matemáticas, transparência referencial (Cap. 15)}
\caltextnext{Programação funcional: recursividade de cauda. (Cap. 15)}
\caltextnext{Programação funcional. (Cap. 15)}
\caltextnext{Programação funcional: funções de ordem superior, composição (Cap. 15)}
\caltextnext{Programação funcional: apply-to-all (map), filter, reduce (Cap. 15)}
\caltextnext{Programação funcional: exemplos em Python e Haskell (Cap. 15)}
\exercicios{Aula de exercícios}
\prova{Prova 3}
%\caltextnext{Apresentação de trabalho}
%\caltextnext{Apresentação de trabalho}
\exercicios{Aula de revisão}
\prova{Reavaliação}
%
% feriados e avisos
% Se n?o quiser que os avisos sejam colocados no cronograma, basta comentar o comando input.
\input{infos/feriados}
%%%avisos 1o semestre
%\aviso{4/13/2016}{Assembleia Ordinária da Adpuc}
\aviso{5/2/2016}{SPEC -- Semana de Palestras da Engenharia de Computação}
\aviso{5/3/2016}{SPEC -- Semana de Palestras da Engenharia de Computação}
\aviso{5/4/2016}{SPEC -- Semana de Palestras da Engenharia de Computação}
\aviso{5/5/2016}{SPEC -- Semana de Palestras da Engenharia de Computação}
\aviso{5/6/2016}{SPEC -- Semana de Palestras da Engenharia de Computação}
\aviso{6/18/2016}{Fechamento do diário}

%avisos 1o semestre
\aviso{8/30/2016}{Seminário de Iniciação Científica -- São Gabriel}

\aviso{8/30/2016}{Simpósio Mineiro de Sistemas de Informação (SMSI) --  Barreiro}
\aviso{8/31/2016}{Simpósio Mineiro de Sistemas de Informação (SMSI) --  Barreiro}
\aviso{9/1/2016}{Simpósio Mineiro de Sistemas de Informação (SMSI) --  Barreiro}

\aviso{8/29/2016}{Semana de Ciência, Arte e Política -- São Gabriel}
\aviso{8/30/2016}{Semana de Ciência, Arte e Política -- São Gabriel}
\aviso{8/31/2016}{Semana de Ciência, Arte e Política -- São Gabriel}
\aviso{9/1/2016}{Semana de Ciência, Arte e Política -- São Gabriel}
\aviso{9/2/2016}{Semana de Ciência, Arte e Política -- São Gabriel}
\aviso{9/3/2016}{Semana de Ciência, Arte e Política -- São Gabriel}

\aviso{9/15/2016}{X Seminário de Extensão da PUC Minas -- Coração Eucarístico}
\aviso{9/16/2016}{X Seminário de Extensão da PUC Minas -- Coração Eucarístico}
\aviso{9/19/2016}{Submissão de projetos ao PROBIC}

\aviso{9/12/2016}{SPEC -- Semana de Palestras da Engenharia de Computação}
\aviso{9/13/2016}{SPEC -- Semana de Palestras da Engenharia de Computação}
\aviso{9/14/2016}{SPEC -- Semana de Palestras da Engenharia de Computação}
\aviso{9/15/2016}{SPEC -- Semana de Palestras da Engenharia de Computação}
\aviso{9/16/2016}{SPEC -- Semana de Palestras da Engenharia de Computação}




\aviso{10/21/2016}{Seminário de Iniciação Científica -- Coração Eucarístico}
\aviso{12/3/2016}{Fechamento do diário}


%

%\aviso{9/19/2022}{AULA ADICIONAL SEGUNDA-FEIRA, DIA 19/09, 10h40}
%\aviso{10/12/2022}{AULA ADICIONAL QUINTA-FEIRA, DIA 13/10, 8h50}
%\aviso{10/31/2022}{AULA ADICIONAL SEGUNDA-FEIRA, DIA 19/09, 10h40}


\end{calendar}
\end{center}

\end{document}
