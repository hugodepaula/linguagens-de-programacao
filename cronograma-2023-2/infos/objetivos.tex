\small
OBJETIVOS GERAIS

- Apresentar um quadro histórico sobre o desenvolvimento das linguagens de programação. 

- Introduzir os conceitos essenciais para a definição, o projeto e o uso das linguagens de programação (estruturas 
sintáticas e semânticas). 

- Apresentar e analisar os diferentes paradigmas de linguagens de programação. 

- Analisar linguagens modernas com foco no uso de tempo e memória das estruturas semânticas existentes. 

- Explorar os mecanismos de concorrência. \\


HABILIDADES A SEREM DESENVOLVIDAS:

- Entendimento da evolução das linguagens de programação, para análise crítica de linguagens utilizadas e de novas 
propostas de linguagens;

- Conhecimento dos princípios de funcionamento das linguagens de programação; 

- Diferenciação de paradigmas de linguagens de programação;

- Entendimento de aspectos relevantes para o projeto de novas linguagens;

- Aplicação dos conceitos aprendidos, tanto na implementação de trabalhos práticos na Universidade, quanto no ambiente 
profissional;

- Capacidade de percepção de integração do conteúdo proposto com outras matérias vistas em disciplinas do Curso.